\documentclass[12pt]{article}
\usepackage[utf8]{inputenc}
\usepackage[spanish]{babel}
\usepackage{amsmath, amssymb}
\usepackage{graphicx}
\usepackage{hyperref}

\title{Optimización de Proteínas en Ciencias de la Computación}
\author{}
\date{}

\begin{document}

\maketitle

\section*{¿Qué es la optimización de proteínas?}

La \textbf{optimización de proteínas} en el área de \textbf{ciencias de la computación} consiste en el uso de métodos computacionales —como algoritmos, simulaciones, inteligencia artificial y modelado matemático— para \textbf{mejorar las propiedades estructurales y funcionales de una proteína}. Este campo es altamente interdisciplinario, combinando conocimientos de biología, química, física, matemáticas e informática.

\section*{Objetivos de la optimización}

Dependiendo del caso, se pueden optimizar diferentes propiedades de una proteína:

\begin{itemize}
    \item \textbf{Estabilidad}: Aumentar su resistencia a condiciones ambientales como temperatura o pH.
    \item \textbf{Actividad}: Incrementar la eficiencia catalítica enzimática.
    \item \textbf{Afinidad de unión}: Mejorar la interacción con otras moléculas (receptores, sustratos, etc.).
    \item \textbf{Especificidad}: Asegurar que la proteína actúe solo sobre blancos específicos.
    \item \textbf{Solubilidad}: Facilitar su producción y manipulación.
\end{itemize}

\section*{Herramientas computacionales}

Las ciencias de la computación proporcionan herramientas clave para abordar problemas complejos de optimización:

\subsection*{1. Modelado y simulación molecular}

Se utilizan técnicas como la \textbf{dinámica molecular} y el \textbf{modelado por homología} para predecir comportamientos estructurales y funcionales de proteínas a nivel atómico.

\subsection*{2. Algoritmos de optimización}

Algoritmos como:

\begin{itemize}
    \item Algoritmos genéticos
    \item Recocido simulado (\textit{Simulated Annealing})
    \item Descenso del gradiente
\end{itemize}

son empleados para encontrar combinaciones óptimas de mutaciones.

\subsection*{3. Aprendizaje automático (\textit{Machine Learning})}

Se entrenan modelos predictivos con datos de proteínas conocidas para estimar propiedades como estabilidad o actividad, y generar nuevas variantes optimizadas.

\subsection*{4. Diseño computacional de proteínas}

Herramientas como \textbf{Rosetta} o \textbf{AlphaFold} permiten diseñar proteínas desde cero o mejorar proteínas existentes mediante simulaciones y predicciones de estructura.

\section*{Aplicaciones}

\begin{itemize}
    \item \textbf{Biotecnología}: Diseño de enzimas más eficientes para procesos industriales.
    \item \textbf{Medicina}: Mejora de anticuerpos o diseño de proteínas terapéuticas.
    \item \textbf{Investigación básica}: Comprensión de la relación entre secuencia, estructura y función proteica.
\end{itemize}

\section*{Ejemplo práctico}

Supongamos que tenemos una enzima capaz de degradar plástico, pero pierde eficiencia a altas temperaturas. Mediante simulaciones computacionales y algoritmos de optimización, es posible evaluar \textbf{virtualmente} miles de mutaciones y predecir cuáles mejoran su \textbf{estabilidad térmica}, reduciendo significativamente el costo y tiempo de experimentación en laboratorio.

\end{document}
