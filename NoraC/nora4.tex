\documentclass[12pt]{article}
\usepackage[utf8]{inputenc}
\usepackage[spanish]{babel}
\usepackage{amsmath, amssymb}
\usepackage{graphicx}
\usepackage{enumitem}
\usepackage{hyperref}
\usepackage{geometry}
\geometry{margin=1in}

\title{\textbf{¿Sigue siendo relevante la Optimización Multiobjetivo (MOO) en la Inteligencia Artificial Moderna?}}
\author{}
\date{}

\begin{document}

\maketitle

\section*{Introducción}

La optimización multiobjetivo (Multi-Objective Optimization, MOO) es una técnica bien establecida para abordar problemas con múltiples funciones objetivo en conflicto. En este informe, se presenta una visión crítica desde el punto de vista de un especialista en \textbf{deep learning} y \textbf{inteligencia artificial moderna}, sobre la vigencia, valor y aplicabilidad actual de MOO.

\section*{¿Es un área activa de investigación?}

Sí. La MOO sigue siendo una \textbf{área activa} a nivel global, tanto en investigación teórica como en aplicaciones prácticas. Existen publicaciones frecuentes en conferencias de alto nivel (GECCO, CEC, IJCAI, NeurIPS) y herramientas modernas como \texttt{PlatEMO}, \texttt{jMetal} o \texttt{Pymoo} que mantienen el interés y evolución del campo.

La MOO ha encontrado nuevos usos en contextos emergentes dentro de la inteligencia artificial, especialmente cuando hay que \textbf{equilibrar múltiples objetivos o restricciones que no pueden fusionarse fácilmente en una sola métrica}.

\section*{¿Es valiosa en deep learning y AI moderna?}

\subsection*{Casos donde \textbf{SÍ} es valiosa}

\begin{itemize}
    \item \textbf{AutoML y NAS (Neural Architecture Search)}: Para optimizar simultáneamente precisión, complejidad, latencia, consumo energético y memoria.
    
    \item \textbf{Compresión y distilación de modelos}: Equilibrando el tamaño del modelo con la pérdida de precisión.
    
    \item \textbf{Sistemas de recomendación y fairness}: Comprometiendo precisión con diversidad, novedad, y criterios éticos.
    
    \item \textbf{Aprendizaje por refuerzo multiobjetivo (MORL)}: Donde un agente tiene múltiples recompensas o metas simultáneas.
    
    \item \textbf{Optimización de hiperparámetros}: Especialmente en contextos de múltiples métricas de evaluación (e.g. precisión vs. sensibilidad).
    
    \item \textbf{Control, robótica y planificación}: Donde múltiples objetivos de seguridad, eficiencia y rendimiento deben considerarse en paralelo.
\end{itemize}

\subsection*{Casos donde \textbf{NO} es tan relevante}

\begin{itemize}
    \item \textbf{Entrenamiento estándar de redes neuronales}: Usualmente se usa una única función de pérdida escalar.
    
    \item \textbf{Problemas con un único KPI dominante}: Como clasificación pura donde solo interesa la \texttt{accuracy}.
    
    \item \textbf{Problemas con baja complejidad de evaluación}: Donde métodos más simples (grid search, optimización bayesiana) son suficientes.
\end{itemize}

\section*{Críticas y limitaciones actuales}

\begin{itemize}ud
    \item \textbf{Poca integración en pipelines de DL}: MOO no forma parte natural del entrenamiento de modelos como sí lo hacen otras técnicas más específicas (dropout, normalization, etc.).
    
    \item \textbf{Complejidad innecesaria}: En muchos casos prácticos, basta con scalarizar múltiples objetivos en una sola función compuesta.
    
    \item \textbf{Escalabilidad limitada}: Muchos algoritmos evolutivos tradicionales no escalan bien a entornos de datos masivos o problemas con alta dimensionalidad.
    
    \item \textbf{Curva de aprendizaje y falta de herramientas}: La aplicación de MOO requiere experiencia adicional y no siempre hay librerías integradas en frameworks de DL (como PyTorch o TensorFlow).
\end{itemize}

\section*{Conclusión}

La optimización multiobjetivo \textbf{sigue siendo una herramienta valiosa y activa}, especialmente en problemas complejos donde los compromisos entre métricas son cruciales. No es la técnica principal para entrenar redes neuronales profundas, pero es cada vez más importante como \textbf{componente auxiliar} en sistemas de inteligencia artificial modernos, particularmente en diseño, optimización, y toma de decisiones realistas.

\vspace{1em}
\noindent
\textbf{Conclusión profesional:} \emph{El verdadero valor de la MOO surge cuando la IA debe funcionar bajo restricciones reales, hacer compromisos éticos, o cumplir múltiples metas simultáneamente. En esos contextos, MOO deja de ser una curiosidad académica y se convierte en una necesidad práctica.}

\end{document}
